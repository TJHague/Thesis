In DIS $F_2$ structure function data, there are four phenomenological regions. In each region, different physics dominates the shape of the structure function ratio. Each kinematic region provides a test bed for our understanding of nuclear physics. Studying these nuclear effects is the driving force behind many experiments \cite{emc_regions,Smirnov99}.

\subsection{Shadowing}

Nuclear shadowing is a phenomenon that occurs in the region of $x<0.1$. Here, there is a depletion of the structure function when compared to deuterium. This depletion increases with mass number $A$. This depletion is also weakly dependent on $Q^2$ and mass number A. This effect is typically explained in terms of nuclear scattering, which can be further investigated in reference \cite{shadowing}.
%at roughly the $\ln Q^2$ scale.

\subsection{Anti-shadowing}

The Anti-shadowing, or enhancement, region is from $0.1 \leq x \leq 0.3$. In this region, the EMC structure function ratios are enhanced to greater than 1. Within experimental uncertainties, there is no $Q^2$ dependence in the anti-shadowing region.

\subsection{EMC Effect}

The EMC effect region spans from $0.3 \leq x \leq 0.8$. In this kinematic area, the EMC structure function ratio falls off and reaches a minimum around $x=0.65$. Since the discovery of the effect by the European Muon Collaboration, extensive EMC effect region data has been recorded over a large $Q^2$ range. The data suggests that the EMC effect is largely independent of $Q^2$. The EMC effect does appear to be logarithmically dependent on mass number $A$.

\subsection{Fermi Motion}

As $x$ is pushed past $0.8$, the EMC structure function ratio sharply increases far beyond unity. In this region, it is known that $F_2^n$, the free nucleon structure function, drops as $\left(1-x\right)^3$. Fermi motion increases the structure function of the bound nucleon, causing the ratio to show this sharp increase.