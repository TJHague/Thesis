There are many theories as to the origin of the EMC effect. To cover them all is beyond the scope of this thesis. This section will discuss the broad classes of models which can be investigated further in \cite{Norton,GST,HenSRC}.

There are two primary groups of theories. Nuclear Structure focuses on the physics of scattering from a nucleus. Nucleon modification focuses on changes to quark momenta due to confinement effects.

\subsection{Nuclear Structure}

\subsubsection{Nucleon Models}

Traditional scattering calculations assume that the scattered nucleon was on-shell. This class of models gives the struck nucleon a negative average energy $\left\langle\epsilon\right\rangle$. This energy shift causes a rescaling of the $x$. This rescaling can explain the EMC effect region and Fermi motion. However, it is not capable of reproducing the anti-shadowing region.

\subsubsection{Pion Enhancement}

An enhancement of the nuclear pion field by nucleon-nucleon interactions. In these models the pion contribution is concentrated to low $x$. The creation of pions also requires the creation of $\Delta$ resonances in the nucleus. 

Alone, this class of model has several problems. To reproduce high $x$ data requires the presence of significantly more $\Delta$s than calculations suggest are plausible. In addition to this, matching anti-shadowing data causes a mismatch in high $x$ data.

\subsection{Nucleon Modification}

\subsubsection{Quark Bags}

In quark cluster models quarks are confined to ``bags'' as defined by the MIT bag model. This creates color-singlet states with multiples of 3 quarks. The most common quark bag models rely on 6-quark bags. 6-quark bags are larger than a nucleon and thus lead to partial deconfinement of the quarks. This increase in confinement size leads to a decrease in quark momenta due to the uncertainty principle. A decrease in quark momenta in this way will suppress the structure function in the EMC region, leading to the EMC effect. \cite{Norton,HenSRC}

This quark bag model alone can compute many nuclear affects. It is hampered by the need for an additional free parameter to compute each new observable. This model has fallen out of favor due to failed predictions in the nuclear Drell-Yan process. \cite{HenSRC}

\subsubsection{Mean Field Enhancement}

Mean Field Enhancement suggests that the structure functions of the nucleons are modified by nucleus surrounding them. Nucleons confined within the nucleus exchange mesons between the quarks of other confined nucleons. This modifies the nucleons structure to change the size of the quark confinement. The predicted increase in confinement size yields a smaller quark momentum. \cite{HenSRC,Daniel_thesis}

\subsubsection{Short Range Correlations}

Short Range Correlations (SRCs) greatly modify a few nucleons, rather than the small modification to all nucleons in mean-field enhancement. SRCs are the idea that there is a probability that two nucleon wavefunctions will overlap. In this scenario the overlapping wavefunctions will cause the size of quark confinement to greatly increase, drastically decreasing the quark momenta.

SRCs also predict an observed high momentum tail at $x>1$. Studies of this effect have noted a correlation between the SRC ``scale factor'' and the strength of the EMC effect, the slope of the EMC ratio in the EMC region. \cite{HenSRC,CLASSRC,WeinsteinSRC}

\subsubsection{Discerning between Mean Field Enhancement and SRCs}

Both Mean Field Enhancement and SRCs have been shown to have accurate predictive power within the datasets available. This leads to the conundrum of finding an unmeasured quantity for which the two models make different predictions. Hen \cite{HenSRC} and Thomas \cite{ThomasSRC} discuss that mean field theory and SRCs make seemingly contradictory predictions for the polarized EMC effect. Mean field theory predicts that polarization will enhance the effect; SRCs predict the polarization to minimize the effect. This will be tested in Jefferson Lab Hall C by measuring the spin structure functions of $^7$Li \cite{CLASspinEMC}.