The EMC effect was first discovered by the European Muon Collaboration (EMC) in 1983. The EMC group measured the structure functions of hydrogen, deuterium, and iron. After correcting for the neutron excess, the per nucleon $F_2$ structure function ratio of iron to deuterium was calculated. This data showed a clear $x$ dependence, contrary to expectations.

Prior to the discovery of the EMC effect, nucleons were assumed to be quasi-free within the nucleus. In this understanding the nuclear $F_2$ structure function would be described as:

\begin{equation}
	F_2^A = ZF_2^p + \left(A-Z\right)F_2^n
\end{equation}

In this understanding, the per nucleon structure function ratio of any two isoscalar targets will be unity. At this time, the only expected nuclear effect was Fermi motion. This effect causes a sharp rise in the per nucleon structure function ratio at high x, but would leave the ratio largely unchanged at low x.

The original experiment did not originally set out to measure the EMC effect, rather the data was a byproduct of efforts to achieve higher luminosity. Because of this, the data had very large uncertainty on the measurement. However, the uncertainties were small enough that the anomaly could be confirmed.

Shortly after the original EMC measurement a Rochester-SLAC-MIT group analyzed previous SLAC data to confirm the phenomenon. The data analyzed not only confirmed the effect in iron, but also in aluminum. The iron data showed the EMC effect as well as the expected rise from Fermi motion at high $x$. The aluminum data showed these phenomena as well as low $x$ shadowing and anti-shadowing. \cite{bodek_Fe,bodek_Al,Norton}

\subsection{Further Results}


Since that time, numerous experiments have measured nuclear $F_2$ structure function data in order to better understand the nature of this anomaly. These searches have primarily focused on heavy isoscalar nuclei.

\textbf{SLAC} At SLAC a new experiment was set with the explicit goal of measuring the EMC effect in a wide range of nuclei. The data cover a large kinematic range of $0.089 \le x \le 0.8$ and $2 \left(\textrm{GeV/c}\right)^2 \le Q^2 \le 15 \left(\textrm{GeV/c}\right)^2$. This data confirms the phenomenon seen by the EMC group in the $x>0.3$ region, but also sees a downturn in the ratios at low x. 

The targets studied were $^4$He, Be, C, Al, Ca, Fe, Ag, and Au. This target range allowed them to study the mass number $A$-dependence of the EMC effect. This data suggests an approximately $\ln A$ dependence on the strength of the EMC effect, with notable outliers of $^4$He and $^9$Be. \cite{Arnold,Gomez}

\textbf{Insert Gomez plot}

%\textbf{Neutrino}

\textbf{BCDMS}The BCDMS experiment at CERN measured the nitrogen and iron EMC ratios. The iron data is consistent with the original EMC measurement within a normalization discrepancy. The nitrogen data is consistent with the SLAC carbon data. However, BCDMS does not demonstrate an $A$-dependence of the EMC effect as SLAC did. This is potentially due to larger uncertainties on the data at high $x$.

\textbf{EMC} The EMC group continued on to do three more experiments to study the EMC effect.

The first of these experiments set out to improve the systematics of the original experiment. NA2' measure the EMC ratio of carbon, copper, and tin. This data agrees with the original data for $x \geq 0.08$. However, below this threshold the NA2' data sees a downturn, the shadowing region, that the original experiment did not see. 

The NA28 experiment focused on studying the EMC ratio of carbon and calcium at low $x$. This data confirms that shadowing does cause the ratio to drop below unity in the region of $x<0.1$. These results also shows that the shadowing region has no $Q^2$ dependence. The data overlaps well with previous measurements.

The last EMC group experiment to study the EMC effect remeasured the copper EMC ratio. To minimize systematics two 1m deuterium targets and three copper targets were used. These results agree with the past results and are of greater precision. %They have been combined with old data???

\textbf{NMC}The New Muon Collaboration (NMC) continued the study of the EMC effect at CERN. Initially NMC measure the EMC ratio of lithium, carbon, and calcium to high precision at low $x$. This data had was taken with two goals: to confirm the EMC data in the shadowing region and to study the effect of nuclear size and density on the EMC ratio. The data confirms the previous EMC measurement and found a very weak $Q^2$ dependence. Lithium and carbon have approximately the same size nucleus, but different nuclear densities. Calcium and carbon have the approximately the same nuclear density, but calcium has a bigger nucleus. It was found that both of these factors play a part in the suppression of the EMC ratio in the shadowing region. Increases in nuclear size or density show an increase in the suppression of the EMC ratio due to nuclear shadowing.

NMC then set out to study the difference between the photonuclear $R$ of different targets in the region of $0.01 \leq x \leq 0.3$. The results were found to, within uncertainties, be compatible with zero. This result confirms that $\nicefrac{\sigma_A}{\sigma_D}=\nicefrac{F_2^A}{F_2^D}$.

Finally, the NMC group studied the EMC effect on Be, C, Al, Ca, Fe, Sn, and Pb. These results again confirm that there is no $Q^2$ dependence of the EMC ratio outside of the shadowing region. These data agree with the SLAC result finding that the EMC effect is approximately logarithmic with $A$.

%\textbf{E665}

\textbf{HERMES}The HERMES experiment ran at the HERA collider. This experiment collided positrons with protons to study the EMC effect on $^3$He and N. Data was taken for $0.013 \le x \le 0.65$ and $0.5 \left(\textrm{GeV/c}\right)^2 \le Q^2 \le 15 \left(\textrm{GeV/c}\right)^2$. In the $x<0.06$ region, the HERMES results differed drastically from the NMC results. Initially this was misreported as an $A$ dependence of photonuclear $R$. This reporting was later amended when it was found that the difference could be attributed to a previously unaccounted for systematic effect.

\textbf{JLab} The E03-103 experiment ran in Hall C at Jefferson Lab. This experiment studied the EMC effect in $^3$He, $^4$He, Be, and C. The kinematics covered were $0.3 < x < 0.9$ and $3 \left(\textrm{GeV/c}\right)^2 < Q^2 < 6 \left(\textrm{GeV/c}\right)^2$. The data measured was not purely DIS, but also included data in the resonance region. This led the experiment to extensively verify that their data was indeed independent of $Q^2$. The data for Beryllium is noted not to match the previous SLAC data. This is caused by the use of a different isoscalar correction and is further rectified by noting normalization uncertainties. These data show a significantly larger EMC effect in Beryllium than expected by the $\ln A$ prediction. The suggested explanation is that the EMC effect is dependent on \textit{local} nuclear density rather than mean nuclear density.