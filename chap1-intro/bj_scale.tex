As $Q^2$ is pushed higher, inelastic scattering begins to give way to DIS. In this kinematic region, the wavelength of the virtual photon is sufficiently short to resolve the internal structure of the nucleon. This transition sees the system begin to behave like a free Dirac particle, the parton. With this in mind, it is useful to look at the cross section for scattering off of a structureless target:

\begin{equation}
	\frac{d^2\sigma}{d\Omega dE^\prime} = \frac{\alpha^2}{4E^{2}\sin^{4}\frac{\theta}{2}} \left[\cos^{2}\frac{\theta}{2} + \frac{Q^2}{2m^2}\sin^{2}\frac{\theta}{2}\right] \delta\left(\nu-\frac{Q^2}{2m}\right)
	\label{xs_no_struct}
\end{equation}

Noting that DIS is scattering off of a structureless parton, we can compare (\ref{xs_inelastic}) and (\ref{xs_no_struct}). By equating these two cross sections, we can clearly extract equations for the DIS structure functions.

\begin{subequations}
\begin{align}
	2mW_1\left(Q^{2},\nu\right) = \frac{Q^2}{2m\nu}\delta\left(1-\frac{Q^2}{2m\nu}\right) \\
	\nu W_2\left(Q^{2},\nu\right) = \delta\left(1-\frac{Q^2}{2m\nu}\right)
\end{align}
\end{subequations}

In this kinematic region, we see that the structure functions are dependent on the ratio $\frac{Q^2}{2m\nu}$ rather than $Q^2$ and $\nu$ independently while the target mass sets the scale. Noting this dependency, the scaling variable Bjorken $x$ is defined. New structure functions, $F_1$ and $F_2$, are also defined in terms of $x$ to clearly show the lack of scaling with $Q^2$ and $\nu$ independently.

\begin{equation}
	x = \frac{Q^2}{2m\nu}
\end{equation}

\begin{subequations}
\begin{align}
	2mW_1\left(Q^{2},\nu\right) \rightarrow F_{1}\left(x\right) \\
	\nu W_2\left(Q^{2},\nu\right) \rightarrow F_{2}\left(x\right)
\end{align}
\end{subequations}

The independence of $F_2$ with respect to $Q^2$ has been experimentally tested. The data was taken at fixed $x$ with varying $Q^2$. All measurements were consistent with no $Q^2$ dependence in the value of $F_2$.

\textbf{Insert plot. Comes from 1972 Ann Rev Nucl Sci 22 (203)}