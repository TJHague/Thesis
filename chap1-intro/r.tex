\textbf{$R=\sigma_L/\sigma_T$. $R$ is independent of atomic number. This facilitates us taking ratios}

If we instead approach DIS as the production and absorption of a virtual photon we can extract a different structure function $R=\sigma_L/\sigma_T$. That is, the ratio of the cross sections for absorbing longitudinal photons to transverse photons.

We can write the DIS cross section in terms of these cross sections as

\begin{equation}
	\frac{d^2\sigma}{d\Omega dE^\prime}\left(E,E^\prime,\theta\right) = \Gamma\left[\sigma_T\left(x,Q^2\right)+\epsilon\sigma_L\left(x,Q^2\right)\right]
\end{equation}

In this equation $\Gamma$ is the flux of transverse virtual photons and $\epsilon$ is the relative flux of longitudinal virtual photons. These are defined by

\begin{equation}
	\Gamma = \frac{\alpha KE^\prime}{2\pi^2Q^2E_0\left(1-\epsilon\right)}
\end{equation}

\begin{equation}
	\epsilon = \frac{1}{1+2\left(1+\nu^2/Q^2\right)\tan^2\left(\frac{\theta}{2}\right)}
\end{equation}

Here, $K$ is the laboratory photon energy

\begin{equation}
	K = \frac{W^2-M^2}{2M}
\end{equation}