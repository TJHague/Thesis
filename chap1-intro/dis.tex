\textbf{How does it work?}
Electron comes in, exchanges photon with nucleon. Boom.

At its most basic, Deep Inelastic Scattering (DIS) is the scattering of a lepton from a nucleon. The two participants exchange a virtual boson, the lepton scatters, and the nucleon is excited to a hadronic final state $X$ with higher mass.

\begin{equation*}
	\ell + N \rightarrow \ell^\prime + X
\end{equation*}

For the MARATHON experiment we will focus on electromagnetic DIS. In this case the lepton is charged and the exchanged virtual boson is a virtual photon. The JLab CEBAF accelerator provides an electron beam, so from here on the lepton will be written as an electron.

\begin{equation*}
	e^- + N \rightarrow e^- + X
\end{equation*}

\textbf{PUT DIS FEYNMAN DIAGRAM HERE}

By interacting with a single nucleon, DIS is a powerful tool for studying nucleon structure. By looking at the DIS cross section, we can see how the nuclear structure functions readily present themselves.

If we assume Lorentz invariance, \textbf{P} and \textbf{T} invariance, and conservation or lepton current, the cross section is

\begin{equation}
	\frac{d^2\sigma}{d\Omega dE^\prime} = \frac{\alpha^2}{Q^4}\frac{E^\prime}{E} L^{\left(s\right)^{\mu\nu}}W^{\left(s\right)}_{\mu\nu}
\end{equation}

where $Q^2$ is the 4-momentum transfer, $E$ is the beam energy, $E^\prime$ is the scattered electron energy, $L^{\left(s\right)^{\mu\nu}}$ is the lepton tensor, and $W^{\left(s\right)}_{\mu\nu}$ is the symmetric hadronic tensor.

When written explicitly in the laboratory frame, we arrive at

\textbf{DIS Cross Section}

\begin{equation}
	\sigma \equiv \frac{d^2\sigma}{d\Omega dE^\prime}\left(E,E^\prime,\theta\right) = \frac{4\alpha^2\left(E^\prime\right)}{Q^4}\cos^2\left(\frac{\theta}{2}\right)\left[\frac{F_2\left(\nu,Q^2\right)}{\nu}+\frac{2F_1\left(\nu,Q^2\right)}{M}\tan^2\left(\frac{\theta}{2}\right)\right]
\end{equation}