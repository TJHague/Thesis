%TO DO:
% - Use something other than x for integral defining structure functions. Using multiple x's is confusing.

Having shown that the nucleons consist of structureless partons we can define physics quantities in terms of the quark-parton model. In this regime, Bjorken $x$ is the fraction of the momentum and energy contained by the parton scattered off of. To access the kinematics of the parton, we simply multiply the energy, momentum, and mass of the nucleon by $x$.

To analyze the structure functions of the nucleon, we first look at elastic scattering off of a parton. In this setup, we imagine that we have the means to determine what parton type the electron was scattered from. This is the same equation as Equation \ref{xs_no_struct}, but with $\alpha$ multiplied by the charge of the parton being scattered from, $e_i$.

\begin{equation}
	\frac{d^2\sigma}{d\Omega dE^\prime} = \frac{\alpha^{2}e_{i}^{2}}{4E^{2}\sin^{4}\frac{\theta}{2}} \left[\cos^{2}\frac{\theta}{2} + \frac{Q^2}{2m^2}\sin^{2}\frac{\theta}{2}\right] \delta\left(\nu-\frac{Q^2}{2m}\right)
\end{equation}

Comparing this with the nuclear inelastic cross section, it is clear that the nuclear structure functions can be written in terms of parton structure functions. The parton structure functions can be derived using the same method as the nuclear structure functions.

\begin{subequations}
\begin{align}
	W_1^i = \frac{e_{i}^{2}Q^2}{4M^{2}x^{2}\nu}\delta\left(1-\frac{Q^2}{2Mx\nu}\right) \\
	W_2^i = \frac{e_{i}^{2}}{\nu}\delta\left(1-\frac{Q^2}{2Mx\nu}\right)
\end{align}
\end{subequations}

If we define $f_{i}\left(x\right)$ as the probability that a parton $i$ has the momentum fraction $x$, or parton distribution, we can then write the nucleon structure functions in terms of the parton structure functions. The delta function makes the integrals trivial.

\begin{subequations}
\begin{align}
	W_{1}\left(Q^{2},\nu\right) = \sum_{i}\int_0^1 \frac{e_{i}^{2}Q^2}{4M^{2}x^{2}\nu}f_{i}\left(x\right)\delta\left(1-\frac{Q^2}{2Mx\nu}\right) dx = \sum_{i} \frac{e_{i}^{2}}{2M}f_{i}\left(x\right) \\
	W_{2}\left(Q^{2},\nu\right) = \sum_{i}\int_0^1 \frac{e_{i}^{2}}{\nu}f_{i}\left(x\right)\delta\left(1-\frac{Q^2}{2Mx\nu}\right) dx = \sum_{i} \frac{e_{i}^{2}}{\nu}f_{i}\left(x\right)
\end{align}
\end{subequations}

Using the definitions of the $F$ structure functions in the previous sections, this formalism allows us to write them in terms of parton quantities.

\begin{subequations}
\begin{align}
	MW_1\left(Q^2,\nu\right) = \sum_i \frac{e_i^2}{2} f_i\left(x\right) \equiv F_1\left(x\right) \\
	\nu W_2\left(Q^2,\nu\right) = \sum_i e_i^2 x f_i\left(x\right) \equiv F_2\left(x\right)
\end{align}
\end{subequations}

From here, the Callan-Gross relation is clear:

\begin{equation}
	F_2\left(x\right) = 2xF_1\left(x\right)
\end{equation}

Utilizing this relation we can write the DIS cross section as:

Deriving the structure functions in terms of parton quantities also allows us to place constraints on the ratio of $F_2$ for the nucleons. Due to mass constraints, we can restrict this analysis to up ($q=2/3$), down ($q=-1/3$), and strange ($q=-1/3$) quarks. Since proton and neutron, along with the up and down quarks, form an isospin doublet we can relate their quark distributions (and extend this to their antiquark distributions):

\begin{subequations}
\begin{align}
	u^p\left(x\right) = d^n\left(x\right) \equiv u \\
	d^p\left(x\right) = u^n\left(x\right) \equiv d \\
	s^p\left(x\right) = s^n\left(x\right) \equiv s 
\end{align}
\end{subequations}

Using these relations, we can write the nucleon structure functions and their ratio as:

\begin{subequations}
\begin{align}
	F_2^p = x\left[\frac{4}{9}\left(u+\bar{u}\right) + \frac{1}{9}\left(d+\bar{d}\right) + \frac{1}{9}\left(s+\bar{s}\right)\right] \\
	F_2^n = x\left[\frac{4}{9}\left(d+\bar{d}\right) + \frac{1}{9}\left(u+\bar{u}\right) + \frac{1}{9}\left(s+\bar{s}\right)\right]
\end{align}
\end{subequations}

\begin{equation}
	\frac{F_2^n}{F_2^p} = \frac{\left[\left(u+\bar{u}\right) + \left(s+\bar{s}\right)\right] + 4\left(d+\bar{d}\right)}{\left[\left(d+\bar{d}\right) + \left(s+\bar{s}\right)\right] + 4\left(u+\bar{u}\right)}
\end{equation}

This equation can be evaluated noting that by definition quark distributions must be positive. This naturally leads to a constraint on $F_2^n/F_2^p$ known as the Nachtmann inequality:

\begin{equation}
	\frac{1}{4} \leq \frac{F_2^n}{F_2^p} \leq 4
\end{equation}

\textbf{INSERT PLOT FROM TALKS SHOWING NACHTMANN INEQUALITY IS SATISFIED}

%"These two form factors, $F_{1,2}(q^2)$, parametrize our ignorance of the detailed structure of the proton". \cite{HaM}
%
%\textbf{TALK MORE ABOUT THIS RELATION}
%
%The following relation allows the cross section to be written in terms of $F_2$ only.
%
%\begin{equation}
%	F_1 = \frac{F_2\left(1+Q^2/\nu^2\right)}{2x\left(1+R\right)}
%\end{equation}
%
%Here $x$ is the bjorken scaling variable and $R$ is the ratio of the longitudinal cross section to the transverse cross section, $\sigma_L/\sigma_T$.
%
%Plugging this in we arrive at
%
%\begin{equation}
%	\frac{d^2\sigma}{d\Omega dE^\prime}\left(E,E^\prime,\theta\right) = \frac{4\alpha^2\left(E^\prime\right)}{Q^4}\cos^2\left(\frac{\theta}{2}\right)F_2\left[\frac{1}{\nu}+\frac{\left(1+Q^2/\nu^2\right)}{xM\left(1+R\right)}\tan^2\left(\frac{\theta}{2}\right)\right]
%\end{equation}