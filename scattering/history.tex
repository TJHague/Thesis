%BRIEF HISTORY

%Rutherford
Ernest Rutherford performed what is often considered the first scattering experiment. This experiment fired an $\alpha$-particle beam at gold foil. The result of this experiment saw most particles pass through the foil completely undeflected. Those that did deflect were scattered at a large range of angles. This gave the world a new view of the nucleus, that of a largely empty space with a few hard scattering centers. We now know that these scattering centers are the nucleus of the atom, formed by a dense combination of protons and neutrons.\cite{GRIF}

Since this time, many experiments have been conducted that expanded our view of the nucleus. The evidence of quarks at SLAC once again revolutionized our understanding of the nucleus. In this experiment, electrons were scattered off protons over a large momentum transfer, $q^2$, and final hadronic invariant mass, $W$, range. This experiment noted a ``surprisingly weak'' $q^2$ dependence once the kinematics reached the $W>2 \textrm{GeV}$ range, a key feature of Deep Inelastic Scattering.\cite{SLAC_DIS} In this view the nucleons, protons and neutrons, are comprised of quarks. These quarks are elementary particles that define the characteristics and structure of the nucleon.\cite{DoQ}

This discovery paved the way for a wave of Deep Inelastic Scattering experiments. These experiments have refined our understanding of the nucleus and its constituent components. Deep Inelastic Scattering has proven to be a one of the most powerful tools available when one seeks to study nuclear structure.

%8-fold way
%As more and more particles were discovered, physicists began looking for an underlying order. In 1961, Murray Gell-Mann and Yuval Ne'eman independently proposed a scheme to explain the abundance of hadrons called SU(3) symmetry. This method groups baryons into categories of spin. From there, using the strangeness and charge an organization scheme appears. 
%
%Gell-Mann designed a graphical representation of these symmetries, which he dubbed the \textit{Eightfold Way}. The spin-0 mesons and spin-\nicefrac{1}{2} octets are each drawn as a hexagon with two particles in the center. The spin-\nicefrac{3}{2} decuplet is drawn as an upside-down triangle. When this method was proposed, all particles in the spin-\nicefrac{3}{2} decuplet were discovered except for the bottom point. Lending credence to this explanation was the prediction, and subsequent discovery, of the $\Omega^-$ particle.
%\cite{DoQ}\cite{GRIF}
%
%%Gell-Mann and Zweig
%%The Quark-Parton Model describes the composition of hadrons, both baryons and mesons. Prior to 1964, it was believed that hadrons were as small as it got. However, the adherence of hadrons to the eight-fold way, a categorization of hadrons by charge and strangeness, suggested that there was some underlying mechanism that had yet to be discovered.
%
%The adherence of hadrons to SU(3) symmetry lacked explanation. In 1964, Gell-Mann and Zweig independently suggested that hadrons could be composed smaller elementary particles. Gell-Mann offered the name ``quarks'' for these constituents. Three quarks were described to be the fundamental building blocks of all known particles: up, down, and strange. These quarks (and their corresponding antiquarks) were purported to be mathematical constructs with fractional charge, \nicefrac{2}{3} for up and \nicefrac{1}{3} for down and strange.
%
%Furthering the theory, in November 1974 two separate experiments published the discovery of the (now named) $J/\psi$ particle. The long lifetime of the $J/\psi$ suggested that new physics must be at play. The Quark-Parton Model predicted the existence of a quark symmetric to the strange quark, called the charm quark. It was determined that the $J/\psi$ could be a meson comprised of a charm and anti-charm, suggesting the validity of the model.