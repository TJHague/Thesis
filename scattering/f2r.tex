The structure functions of the nucleons are common inputs to models. Making use of a Hydrogen target the $F_2^p$ structure function is easily accessible. Unfortunately, there is no free neutron target. This absence means that there is no way to direct way to measure $F_2^n$. However, with the proper input, we can extract the ratio $F_2^n/F_2^p$.

To extract this ratio, we first define ``EMC-type'' ratios. These are simply the ratio of the nuclear structure function to the sum of its constituent nucleons. The ``EMC-type'' ratios for $^3$He and $^2$H are:

\begin{equation}
	R_{\text{EMC}}\left(^3\text{He}\right) = \frac{F_2^{^3\text{He}}}{2F_2^p + F_2^n}
\end{equation}

\begin{equation}
	R_{\text{EMC}}\left(^2\text{H}\right) = \frac{F_2^{^2\text{H}}}{F_2^p + F_2^n}
\end{equation}

These can be used to create a ``Super-Ratio'', $\mathcal{R}$, as the ratio of ``EMC-type'' ratios.

\begin{equation}
	\mathcal{R} = \frac{R_{\text{EMC}}\left(^3\text{He}\right)}{R_{\text{EMC}}\left(^2\text{H}\right)} = \frac{F_2^{^3\text{He}}}{2F_2^p + F_2^n} \cdot \frac{F_2^p + F_2^n}{F_2^{^2\text{H}}}
\end{equation}

Solving this for $F_2^n/F_2^p$ makes it clear that the quantity can be easily extracted with a cross section ratio measurement and a model input for $\mathcal{R}$.

\begin{equation}
	\frac{F_2^n}{F_2^p} = \frac{F_2^{^3\text{He}}/F_2^{^2\text{H}} - 2\mathcal{R}}{\mathcal{R} - F_2^{^3\text{He}}/F_2^{^2\text{H}}}
\end{equation}