\section{Helium-3/Deuterium Yield Ratio}

Using the analysis method outlined in Section \ref{sec:analysis_outline}, the Helium-3/Deuterium yield ratio was calculated. The data is shown in Figure and listed in Table.

\section{Helium-3 EMC Yield Ratio}

The Helium-3 EMC ratio is the Helium-3/Deuterium ratio with Helium-3 corrected for proton-excess. That is, Helium-3 is transformed into a hypothetical isoscalar A=$\nicefrac{3}{2}$ nucleus. The isoscalar correction is done using the method described in Section \ref{sec:isocor}. This analysis uses an input of an $F_2^n/F_2^p$ fit of the extraction from the $^3H/^3He$ MARATHON data. This data in shown in Figure and listed in Table.

\section{Normalizing the data}

A feature of all EMC data is a unity crossing near $x=0.3$. This leads to the understanding that there are minimal nuclear effects in the region near $x=0.3$. A look at Figure will show that this feature is not present in this data. An interpretation of this, as this work will examine, is that the Helium-3 data is in need of normalization. This is further justified when reading Reference \textbf{Tong's Thesis} and it is noted that $^3H/^3He$ required a $2.8\%$ normalization in order for the extraction of $\nicefrac{F_2^n}{F_2^p}$ to agree with the extraction from $^2H/^1H$. The $^2H/^1H$ data is in agreement with world data.

The normalization for this data is determined in the same way that it was determined for the $^3H/^3He$ data. This is done by first extracting $\nicefrac{F_2^n}{F_2^p}$ from the Helium-3/Deuterium Yield Ratio using the method described in Section \label{sec:F2ratio}. Since nuclear effects are minimal near $x=0.3$, it is expected that all $F_2^n/F_2^p$ extractions should agree in this region. Figure \textbf{blah} shows that, without normalization, this is not the case.

This extraction requires a model input. For this analysis, the Kulagin-Petti (KP) model is used. This model was chosen by comparing it to the non-isoscalar yield ratio. Of the models examined, this one best matched the overall shape of the data. This matching is shown in Figure \textbf{blah}.

\textbf{This is compared to?????} Figure clearly shows a lack of agreement in this region.

The $\chi^2$ of the comparison of these points is then calculated. Different normalizations of the yield ratio are then iterated over in steps of $0.1\%$. This is continued until the extractions are clearly deviating again and the normalization with the minimum $\chi^2$ is chosen. For this data, the normalization is \textbf{blah}. Figure \textbf{blah} shows the $\nicefrac{F_2^n}{F_2^p}$ extraction with this normalization applied. Figure \textbf{blah} shows the Helium-3/Deuterium yield ratio and isoscalar Helium-3 EMC ratio with this this normalization applied. Table \textbf{blah} lists the data with the normalization applied.

\section{Comparison to Previous Helium-3 EMC Data}



\section{Analyzing the EMC slope}

Studies of the EMC effect often strive to look for correlations between the strength of the EMC effect and other nuclear quantities. To do this, a measure of the EMC strength must be defined. The typical definition of this is the absolute value of the slope of the isoscalar EMC ratio in the range $0.35 \leq x \leq 0.7$, referred to as $\left|dR_{\text{EMC}}/dx\right|$. One benefit of this definition is that it is largely free of normalization uncertainties. 

Figure \textbf{blah} shows this fit for the Helium-3 EMC ratio. In selecting the data for the fit, the cut was placed, as defined in the previous paragraph, on the range of $0.35 \leq x \leq 0.7$. This has the effect of omitting the bins centered at $x=0.345$ and $x=0.705$. Both of these bins, while centered outside of the range, contain data within the range. The decision was made to exclude these points in order to maintain consistency with the traditional extraction that only uses data in the defined range. The effect of the inclusion of these points have on the slope extraction was examined. The inclusion of the $x=0.345$ bin resulted in an approximately $5\%$ lowering of the slope. This small a small effect because the bins value is nearly centered within the statistical fluctuations in the data and the EMC slopes are often nearly linear down to approximately $x=0.25$. Inclusion of the $x=0.705$ bin resulted in a nearly $50\%$ decrease in the slope. This is a significantly larger effect. This is because the bin at $x=0.705$ not only appears to have a very large fluctuation from the trend of the data, but this is also the region where Fermi motion begins to flatten out the EMC slope until the data makes a sharp turn upward. The exclusion of these points means that the fit data is in the range of $0.36 \leq x \leq 0.69$.

Here the data is shown along with other EMC data to study correlations with nuclear quantities. These data are from the SLAC E139, JLab E03-103, and CLAS experiments. Extractions of the EMC slopes from these data have been completed by both Arrington \textit{et. al.} \cite{arrington_src} and Malace \textit{et. al.} \cite{malace_emc}. Each of these experiments and extractions use a different isoscalar correction, which has an effect on the extracted slope. In order to ensure that correlations are properly examined, isoscalar corrections should be applied in a consistent way to all data. For the following plots, the data from each of these experiments has the MARATHON $\nicefrac{F_2^n}{F_2^p}$ based isoscalar correction applied. The EMC slopes are then extracted by a linear fit of the data in the $0.35 \leq x \leq 0.7$ region. Due to binning, the range of data included for E139 is $0.36 \leq x \leq 0.68$. Due to binning and available data, the range of data included for CLAS is $0.353 \leq x \leq 0.58$. Tables \textbf{blah} show the slopes extracted in this method from each experiment.

The two most commonly studied correlations are with mass number $A$ and scaled nuclear density. Nuclear density, measured in nucleons/fm$^{-3}$, is the number of nucleons per unit volume of the nucleus. In this analysis, the nuclear density is calculated using the hard sphere approximation, $\rho\left(A\right) = 3A/4\pi R^3$. In this equation, $R^2=5\left<r^2\right>/3$ where $\left<r^2\right>$ is the rms charge radius of the nucleus being studied. The rms charge radii used come from \cite{DeVries} which are extracted from electron scattering experiments. In cases where multiple values are listed, the average of the values are used. The exception to this is silver-108 (Ag108), as charge radius data for Ag108 is unavailable. Nuclear charge radius is correlated with $A$, so to approximate the charge radius of Ag108 the charge radii of Ag107 and Ag109 from \cite{2012_charge_radii} were averaged. Scaled nuclear density is the nuclear density scaled by a factor of $\left(\text{A}-1\right)$/A. This scaling removes the struck nucleons contribution to the density from the value. The correlation of the EMC effect with this value has been noticed in many past experiments, as described in Chapter \ref{chap:emc}. Figures \textbf{blah} show these correlations.

\cite{slope_predict} examines the correlation of the EMC slope with nuclear binding energy per nucleon and the residual strong interaction energy (RSIE) per nucleon. Nuclear binding is typically understood to play a part in the EMC effect, but is considered insufficient to completely explain the effect. The RSIE is defined as the energy loss of nucleons binding together by strong interaction. RSIE is calculated by removing the Coulomb contribution to the binding energy of the nucleus. That is, $RSIE\left(A,Z\right) = B\left(A,Z\right) + \left(0.71 \text{MeV}\right)Z\left(Z-1\right)A^{-\nicefrac{1}{3}}$ where $B\left(A,Z\right)$ is the binding energy of the nucleus. This calculation assumes that nuclear binding is only comprised of strong and electro-magnetic interactions. For these figures and calculations, the binding energies from \cite{BindingEnergy} are used. Figures \textbf{blah} show these correlations.

Another correlation often studied is with the average nucleon separation energy, $\left<\epsilon\right>$. This value is another means of examining the nuclear binding model in the context of the EMC effect. In this scheme, nuclear binding causes the nucleons to have a level of ``off-shellness''. The separation energy of a nucleon is a measure of how off-shell the nucleon is. The separation energy needed causes a rescaling of $x$ by approximately $\left<\epsilon\right>/m$, where $m$ is the mass of the nucleon. Nucleon separation energy is at the heart of models that include off-shell corrections. Off-shell corrections have been shown to well-approximate the shape of the EMC effect, though they cannot describe the effect on their own. The $\left<\epsilon\right>$ values used here were obtained from \cite{arrington_src}. Figure \textbf{blah} shows this correlation. The data for Lead-208 is unavailable and is thus excluded from this figure.

The final correlation looked at is with the Short-Range Correlation Scaling Coefficient, $a_2$. In the SRC model, correlated pairs of nucleons are greatly modified giving rise to the EMC effect. A measure of the probability that a nucleon belongs to a nucleon-nucleon SRC pair can be obtained by measuring $a_2$. $a_2$ is the height of a plateau observed when studying the per-nucleon cross section ratio of a nuclear target to that of deuterium in the $Q^2 > 1.4 \text{GeV}^2$ and $1.5 \leq x \leq 1.9$ range. The $a_2$ values used were obtained from \cite{arrington_src} as it has the most complete set, this ensures that the extractions were treated accordingly. Lead-208 data is only available in \cite{clas_emc} and is used for this figure. Figure \textbf{blah} shows this correlation.