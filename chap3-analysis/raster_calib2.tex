The raster is calibrated by defining a line that maps the raster current to positions at each BPM and the target. To do this, the slope and intercept of this line had to be determined. The slope corresponds to the conversion of raster current to position displacement. The intercept is then determined from the central position that the beam is displaced from. This section will be a general presentation of the techniques used to calibrate the raster. For a more in-depth discussion of how the raster was calibrated, see Appendix \ref{raster_appendix}.

For the horizontal raster, this was done by optimizing the reconstructed z-vertex on the target. When properly calibrated, there should be no correlation between the horizontal raster and the z-vertex. Linear interpolation between two ``bad'' calibrations is a simple way to determine the correct calibration slope.

The veritcal raster could be calibrated in a similar way by minimizing the correlation between the vertical raster and a known momentum phenomena (i.e. a $W^2$ peak). Unfortunately, such a feature does not exist within the kinematics of our data. The vertical calibration was determined using the carbon hole target. The hole is known to be $2mm$ diameter. By using the raster data, the hole can be fit in order to determine the vertical calibration slope.

The intercepts are determined by looking at the mean BPM position readings and projecting these to the target. This position will correspond to the mean value of the rasters as well. Using the beam position, raster current, and calibration slope the calibration intercept can easily be determined.