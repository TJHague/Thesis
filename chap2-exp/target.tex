%Put this in order they appear in beamline?
\section{Tritium Target System}

\subsection{Cell Design}

\subsection{Target Chamber and Ladder}
The MARATHON experiment used six different targets for physics. These are
\begin{itemize}
	\item Tritium
	\item Deuterium
	\item Hydrogen
	\item Helium-3
	\item Empty Cell
	\item 25cm Dummy
\end{itemize}

With the exception of the 25cm Dummy target, all of these targets use the same cell design. The target cells are 25 centimeters long and made of Aluminum 7075. The cells are sealed and utilize conductive cooling to mitigate the risk of a Tritium leak.

In a closed cell, localized beam heating can create a "boiling" effect on the gas. This is seen as a change in the effective density of the target. Experimental time was dedicated to understanding the magnitude of this effect and will be further discussed in section blah.

%Put in target data table: endcap thicknesses, density, fill pressure, target thickness

The 25cm Dummy is comprised of two Aluminum 7075 foils 25cm apart, this is the same material as the other target cells. Each foil is $0.3495\pm0.0006$ $g/cm^2$ thick, significantly thicker than the cell walls.

In addition to these, we also utilized several solid targets for procedural studies. %procedural is a stupid word. fix it.
\begin{itemize}
	\item Optics Target - 11 Carbon Foils
	\item Carbon Hole - A carbon foil with a 2mm diameter hole in the center
	\item Raster Target - A "straw" for ensuring the beam is not coming in at an angle
	\item Thick Aluminum - For calibrating the ion chambers
	\item Single Carbon Foil
	\item Titanium
	\item Beryllium Oxide
\end{itemize}