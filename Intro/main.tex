%Prior to \textbf{EMC EFFECT STUFF} it was taken for granted that quark distributions were independent of the surrounding nucleus. The European Muon Collaboration (EMC) found this not to be the case. They found that quark distributions are different in a nucleus than in a nucleon. The ratio of the Iron to Deuterium $F_2$ structure functions was found to greatly deviate from unity.
%
%EMC EMC EMC. $A=3$ nuclear structure data is critical to understanding the nature of the EMC effect.
%
%The MARATHON experiment studied the cross section ratios of the He3/H2, H3/H2, H3/He3. This data aims to provide a more complete understanding of the EMC effect and nuclear structure. In this thesis, I will describe the analysis of the He3 EMC effect. The next two chapters will provide an overview of electron-nucleon Deep Inelastic Scattering and the EMC effect. Chapter 4 will describe the experimental setup used in Hall A of Thomas Jefferson National Accelerator Facility (JLab). Chapter 5 will go over the analysis of the cross section ratio. Finally, Chapter 6 will present the results of my analysis.

%%%%%%

A pioneering experiment at SLAC \cite{SLAC_DIS} showed that, at sufficiently high momentum transfer, lepton scattering is sensitive to the partonic structure of the nucleon. This led to a renaissance of scattering experiments, all vying to better our understanding of nuclear structure and the constituent parts that make up the nucleus. High momentum transfer scattering, dubbed Deep Inelastic Scattering (DIS), has proven to be an invaluable tool for the study of nuclear physics.

One puzzle still unsolved, born out of the using DIS to study the nuclear structure functions, is that of the EMC effect. The European Muon Collaboration, the namesake of the EMC effect, studied the ratio of the per nucleon cross-sections of Iron (corrected for neutron excess) to Deuterium in an effort to understand their experimental systematics \cite{emc_FE}. What was found was clear evidence for nucleon modification within the nucleus. Where they expected a measurement of unity, the ratio exhibited a downward slope in the region of the data. This is the EMC effect.

Since then, many experiments have contributed data sets to the study of the EMC effect. These data have shown correlations between the strength of the EMC effect and various nuclear quantities such as mass number $A$, nuclear density, and short range correlations within the nucleus. The available measurements have primarily focused on heavy nuclei, with the exception of the JLab E03-103 experiment.

The MARATHON experiment ran in Hall A of Thomas Jefferson National Accelerator Facility (JLab) using a $10.59\ \text{GeV}$ electron beam from the CEBAF accelerator. One of the primary goals for MARATHON was to measure the EMC effect in the $A=3$ mirror nuclei, $^3He$ and $^3H$ \cite{proposal}. The measurement of the $A=3$ EMC effects are considered critical to a more complete understanding of the EMC effect. This thesis will present the study of the $^3He$ EMC effect from the MARATHON data. Chapter \ref{chap:scattering} will give an overview of electron scattering with a focus on Deep Inelastic Scattering. Chapter \ref{chap:emc} will discuss the history of the EMC effect as well as a selection of models that aim to describe it. Chapter \ref{chap:setup} describes the experimental setup of the MARATHON experiment at JLab. Chapter \ref{chap:analysis} shows the methods used to analyze the data acquired from the experiment. Finally, Chapter \ref{chap:results} presents the results of this study as well as a look into how the measured strength of the $^3He$ EMC effect fits in with correlations that are useful for understanding the source of this effect.