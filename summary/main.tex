The MARATHON experiment measured the Helium-3/Deuterium yield ratio in the region of $0.195 \leq x \leq 0.825$, $2.5 \left(\text{Gev}/c\right)^2 \leq Q^2 \leq 13 \left(\text{Gev}/c\right)^2$, and $3.5 \left(\text{Gev}/c^2\right)^2 \leq W^2 \leq 13 \left(\text{Gev}/c^2\right)^2$. These data were collected in Hall A at Thomas Jefferson National Accelerator Facility (JLab) utilizing both standard equipment High-Resolution Spectrometers (HRSs). The Left HRS had a momentum setting of $3.1\ \text{GeV}/c$ and measured scattered electrons over an angular range of $17\degree$-$33.5\degree$. The Right HRS had a momentum setting of $2.9\ \text{GeV}/c$ and measured scattered electrons at an angle of $36\degree$. The CEBAF accelerator delivered $10.59 \text{GeV}$ electrons incident on the target at $22.5\mu A$. This is the first measurement of the Helium-3 EMC effect purely in the Deep-Inelastic Scattering region. This data is in good agreement with previous Helium-3 EMC measurements, which included data in the resonance region, and low-$x$ Helium-3/Deuterium data.

A study was done comparing the $\nicefrac{F_2^n}{F_2^p}$ extraction from the Helium-3/Deuterium yield ratio to the extraction from MARATHON $^2H/^1H$ data. The $\nicefrac{F_2^n}{F_2^p}$ extraction from $^2H/^1H$ agrees well with world data. An agreement between these extractions was expected in the vicinity of $x=0.3$, which was absent in our data. This analysis of the data applies a $2.8\%$ normalization to the $^3He$ data in order to bring the extractions into agreement. An explanation for this discrepancy has not been determined. Hypotheses suggested include that the understanding of nuclear effects in $^3He$ are more poorly understood than expected or that there is an isovector dependent EMC effect.

The Helium-3 EMC Ratio was corrected for proton excess utilizing the $\nicefrac{F_2^n}{F_2^p}$ extracted from MARATHON $^3H/^3He$ data. This yields EMC data for a hypothetical $A=3$, $Z=\nicefrac{3}{2}$ nucleus. The strength of the EMC effect is measured by the slope of the isoscalar data in the region of $0.35 \leq x \leq 0.7$. The strength of the Helium-3 EMC effect is shown alongside the strength of the EMC effect from various other nuclei from past experiments with the MARATHON isoscalar correction applied. This addition of data is consistent with prior analyses of EMC effect correlations.

This measurement provides critical data to help create a complete picture of the EMC effect. While we certainly have a better understanding of the EMC effect, the puzzle is far from solved. Two leading explanations for the EMC effect are Mean Field Enhancement and Short Range Correlations (SRC). These two theories have proven to have very accurate predictive power. The next frontier in studying the EMC effect is to make a measurement where the predictions by these models diverge. Mean field enhancement predicts that polarization will enhance the EMC effect while SRCs predict polarization will minimize the effect. CLAS in Hall C at JLab will measure the spin structure functions of $^7$Li in order to determine its polarized EMC effect.